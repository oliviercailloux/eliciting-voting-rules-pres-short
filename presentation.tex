\RequirePackage[l2tabu, orthodox]{nag}
\RequirePackage{silence}
\WarningFilter{nag}{There is no environment ``centering'' }%nag complains because beamer titlepage uses a centering environment
\WarningFilter{nag}{1 complaints in total}
\WarningFilter{ifpdf}{Someone has redefined \pdfoutput}
\WarningFilter{fmtcount}{\ordinal already defined use \FCordinal instead}
\documentclass[english]{beamer}
%INSTALL

%avoids a warning
\usepackage[log-declarations=false]{xparse}
%Should be inputed before xunicode, someone says. And put everything font-related before fontspec. But I don’t need it anyway.
% \usepackage{amssymb} %for e.g. succcurlyeq.
\usepackage{fontspec} %font selecting commands
%Call amssymb before amsmath or any other amslatex style files (amsthm, ...).
\usepackage{xunicode}
%warn about missing characters
\tracinglostchars=2

%REDAC
\usepackage{booktabs}
\usepackage{cleveref}
\usepackage{calc}

\usepackage{mathtools} %load this before babel!
	\mathtoolsset{showonlyrefs,showmanualtags}

\usepackage{babel}

\usepackage[super]{nth}
\usepackage{listings} %typeset source code listings
	\lstset{language=XML,tabsize=2,literate={"}{{\tt"}}1,captionpos=b}
\usepackage[nolist,smaller,printonlyused]{acronym}
\usepackage[nodayofweek]{datetime}%must be loaded after the babel package
\usepackage{xspace}
\usepackage{hyperref}
%breaklinks makes links on multiple lines into different PDF links to the same target.
%colorlinks (false): Colors the text of links and anchors. The colors chosen depend on the the type of link. In spite of colored boxes, the colored text remains when printing.
%linkcolor=black: this leaves other links in colors, e.g. refs in green, don't print well.
%pdfborder (0 0 1, set to 0 0 0 if colorlinks): width of PDF link border
%hidelinks
\hypersetup{breaklinks,bookmarksopen}
% hyperref doc says: Package bookmark replaces hyperref’s bookmark organization by a new algorithm (...) Therefore I recommend using this package.
\usepackage{bookmark}

% center floats by default, but do not use with float
% \usepackage{floatrow}
% \makeatletter
% \g@addto@macro\@floatboxreset\centering
% \makeatother
\usepackage{ragged2e} %new com­mands \Cen­ter­ing, \RaggedLeft, and \RaggedRight and new en­vi­ron­ments Cen­ter, FlushLeft, and FlushRight, which set ragged text and are eas­ily con­fig­urable to al­low hy­phen­ation (the cor­re­spond­ing com­mands in LaTeX, all of whose names are lower-case, pre­vent hy­phen­ation al­to­gether). 
\usepackage{siunitx} %[expproduct=tighttimes, decimalsymbol=comma]
\sisetup{detect-all}% to detect e.g. when in math mode (use a math font)
\usepackage{braket} %for \Set
\usepackage{natbib}

\usepackage{amsmath,amsthm}
% \usepackage{amsfonts} %not required?
% \usepackage{dsfont} %for what?
%unicode-math overwrites the following commands from the mathtools package: \dblcolon, \coloneqq, \Coloneqq, \eqqcolon. Using the other colon-like commands from mathtools will lead to inconsistencies. Plus, Using \overbracket and \underbracke from mathtools package. Use \Uoverbracket and \Uunderbracke for original unicode-math definition.
%use exclusively \mathbf and choose math bold style below.
\usepackage[warnings-off={mathtools-colon, mathtools-overbracket}, bold-style=ISO]{unicode-math}

%\setmainfont{Latin Modern Roman}%sans, main, mono all default to Latin Modern family.
%\setsansfont{Latin Modern Sans}
%\setmonofont{}

\newfontfamily\xitsfamily[
	BoldFont = xits-bold.otf,%
	ItalicFont = xits-italic.otf,%
	BoldItalicFont = xits-bolditalic.otf,%
	Mapping=tex-text% to turn "--" into dashes, useful for bibtex%%
]{xits-regular.otf}

\setmainfont[%
	Fractions=On,
	BoldFont = xits-bold.otf,%
	ItalicFont = xits-italic.otf,%
	BoldItalicFont = xits-bolditalic.otf,%
	Mapping=tex-text% to turn "--" into dashes, useful for bibtex%
]{xits-regular.otf}
\setmathfont{xits-math.otf}

% setminus is missing in lmmath font and in texgyrepagella-math
%\AtBeginDocument{\renewcommand{\setminus}{\smallsetminus}}
%\setmathfont[range={"29F5}]{xits-math.otf}
%∉
%\setmathfont[range={"2209}]{texgyrepagella-math.otf}
%mapsto
%\setmathfont[range={"21A6}]{texgyrepagella-math.otf}
%✓
%\setmathfont[range={"2713}]{texgyrepagella-math.otf}
%\vert, symbol code found in unimath-symbols.pdf (also used for \lvert and \rvert)
%\setmathfont[range={"007C}]{texgyrepagella-math.otf}

%GRAPHICS
\usepackage{pgf}
\usepackage{pgfplots}
	\usetikzlibrary{matrix,fit,plotmarks,calc,trees,shapes.geometric,positioning,plothandlers}
\usepackage{graphicx}

\graphicspath{{graphics/},{graphics-dm/}}
\DeclareGraphicsExtensions{.pdf}

%HACKING
\usepackage{printlen}
\uselengthunit{mm}
% 	\newlength{\templ}% or LenTemp?
% 	\setlength{\templ}{6 pt}
% 	\printlength{\templ}
\usepackage{etoolbox} %for addtocmd command
\usepackage{scrhack}% load at end. Corrects a bug in float package, which is outdated but might be used by other packages
\usepackage{xltxtra} %somebody said that this is loaded by fontspec, but does not seem correct: if not loaded explicitly, does not appear in the log and \showhyphens is not corrected.

%Beamer-specific
 \usepackage{appendixnumberbeamer}
% \newcommand{\citep}{\cite}%Better: leave natbib.
\setbeamersize{text margin left=0.8cm, text margin right=0.8cm} 
\setbeamertemplate{navigation symbols}{} 
\usetheme{BrusselsBelgium}
\usefonttheme{professionalfonts}

\newcommand{\R}{ℝ}
\newcommand{\N}{ℕ}
\newcommand{\Z}{ℤ}
\newcommand{\card}[1]{\lvert{#1}\rvert}
\newcommand{\powerset}[1]{\mathcal{P}(#1)}
\newcommand{\suchthat}{\;\ifnum\currentgrouptype=16 \middle\fi|\;}
%\newcommand{\Rplus}{\reels^+\xspace}

\AtBeginDocument{%
	\renewcommand{\epsilon}{\varepsilon}
% we want straight form of \phi for mathematics, as recommended in UTR #25: Unicode support for mathematics.
%	\renewcommand{\phi}{\varphi}
}

% with amssymb, but I don’t want to use amssymb just for that.
% \newcommand{\restr}[2]{{#1}_{\restriction #2}}
\newcommand{\restr}[2]{{#1}_{\vert #2}}
\DeclareMathOperator*{\argmax}{arg\,max}


\newcommand{\allalts}{\mathcal{A}}
\newcommand{\alts}{A}
\newcommand{\allrules}{\mathcal{F}}

%the set of all profiles
\newcommand{\profiles}{\mathbfcal{R}}
%a rank-profile
\newcommand{\rprof}{\mathbf{x}}
% \newcommand{\rprof}{\text{\boldmath $x$}
\newcommand{\rankprofilesrvs}{\mathcal{X}}

% constraints over ranks
\newcommand{\prrefceq}{≥^C}
\newcommand{\prrefc}{>^C}
\newcommand{\prrefcsim}{\sim^C}

\newcommand{\wos}{\mathcal{W}}
\newcommand{\prs}{\mathcal{Z}}
%preorder induced by consistent C
\newcommand{\prCeq}{\ppglobeq^C}%\framebox{C}
\newcommand{\prC}{\ppglob^C}
\newcommand{\halfvoters}{\frac{n}{2}}
% dominates strictly
% \newcommand{\doms}{\RHD}
\newcommand{\doms}{\mathrel{\blacktriangleright}}
% \newcommand{\rvspace}{\mathbb{X}}
\newcommand{\rvspace}{\left[1, m\right]^N}
\newcommand{\cspace}{\mathcal{C}}
% pref over ranks (individually)
\newcommand{\prrefeq}{\succcurlyeq}
\newcommand{\prreff}{\succ}
% global weak-order over ranks
\newcommand{\pglobeq}{⪰}
%that’s not really what I want, but \not\succsim crashes xelatex
\def\not\pglobeq{\nsucceq}
%\newcommand{\pglob}{⪲}
\newcommand{\pglob}{⪲}
% global (partial) preorder over ranks
\newcommand{\ppglobeq}{\succsim}
\newcommand{\ppglobsim}{\sim}
\newcommand{\ppglob}{\succnsim}
%\newcommand{\ppglob}{\succ}
% \def\not\ppglobeq{\nsucceq}
%U+22E1
% \def\not\ppglobeq{⋡}
%U+2275
\def\not\ppglobeq{≵}

\newcommand{\Vpr}{V^{\,\prime}}

%Voting and MCDA
%\newcommand{\allalts}{\mathscr{A}}

%Voting
%rank-profile voting rule
\newcommand{\frp}{f_\text{r-p}}
\newcommand{\feasalts}{F}
\newcommand{\allvoters}{\mathscr{N}}
\newcommand{\voters}{N}
\newcommand{\allsystems}{\mathcal{G}}
\newcommand{\prof}{\mathbf{R}}
\newcommand{\allprofs}{\mathbfcal{R}}
\newcommand{\linors}{\mathcal{L}(\alts)}

\newcommand{\pbasic}[1]{\prof^{#1}_\epsilon}
\newcommand{\pelem}[1]{\prof^{#1}_e}
\newcommand{\pcycl}[1]{\prof^{#1}_c}
\newcommand{\pcycllong}[1]{\prof^{#1}_{cl}}
\newcommand{\pinv}[1]{\overline{\prof_{#1}}}
\newcommand{\dmap}{{\xitsfamily δ}}
%powerset without zero
\newcommand{\powersetz}[1]{\mathcal{P}_\emptyset(#1)}

%logic atom
%⟼ (long)
\DeclareDocumentCommand{\lato}{ O{\prof} O{\alts} }{[#1 \!⟼\! #2]}
%logic atom in
%↝, \stackrel{\in}{\mapsto}, ➲, ⥹
\newcommand{\tightoverset}[2]{%
  \mathop{#2}\limits^{\vbox to -.5ex{\kern-0.9ex\hbox{$#1$}\vss}}}
\DeclareDocumentCommand{\latoin}{ O{\prof} O{\alpha} }{[#1 \tightoverset{\in}{⟼} #2]}
\newcommand{\alllang}{\mathcal{L}}
\newcommand{\ltru}{\texttt{T}}
\newcommand{\lfal}{\texttt{F}}
\newcommand{\laxiom}[1]{{\texgyreherosfamily{\textsc{#1}}}}

%ARG TH
\newcommand{\AF}{\mathcal{AF}}
\newcommand{\labelling}{\mathcal{L}}
\newcommand{\labin}{\textbf{in}\xspace}
\newcommand{\labout}{\textbf{out}}
\newcommand{\labund}{\textbf{undec}\xspace}
\newcommand{\nonemptyor}[2]{\ifthenelse{\equal{#1}{}}{#2}{#1}}
\newcommand{\gextlab}[2][]{
	\labelling{\mathcal{GE}}_{(#2, \nonemptyor{#1}{\ibeatsr{#2}})}
}
\newcommand{\allargs}{A^*}
\newcommand{\args}{A}
\newcommand{\ar}{a}
\newcommand{\ext}{\mathcal{E}}

%MCDA+Arg
\newcommand{\dm}{d}
\newcommand{\ileadsto}{\rightcurvedarrow}
\newcommand{\mleadsto}[1][\eta]{\rightcurvedarrow_{#1}}
\newcommand{\ibeats}{\vartriangleright}
\newcommand{\mbeats}[1][\eta]{\vartriangleright_{#1}}

%MISC
\newcommand{\lequiv}{\Vvdash}
\newcommand{\weightst}{W^{\,t}}

%MCDA classical
\newcommand{\crits}{\mathcal{J}}

%Sorting
\newcommand{\cats}{\mathcal{C}}
\newcommand{\catssubsets}{2^\cats}
\newcommand{\catgg}{\vartriangleright}
\newcommand{\catll}{\vartriangleleft}
\newcommand{\catleq}{\trianglelefteq}
\newcommand{\catgeq}{\trianglerighteq}
\newcommand{\alttoc}[2][x]{(#1 \xrightarrow{} #2)}
\newcommand{\alttocat}[3]{(#2 \xrightarrow{#1} #3)}
\newcommand{\alttoI}{(x \xrightarrow{} \left[\underline{C_x}, \overline{C_x}\right])}
\newcommand{\alttocatdm}[3][t]{\left(#2 \thinspace \raisebox{-3pt}{$\xrightarrow{#1}$}\thinspace #3\right)}
\newcommand{\alttocatatleast}[2]{\left(#1 \thinspace \raisebox{-3pt}{$\xrightarrow[]{≥}$}\thinspace #2\right)}
\newcommand{\alttocatatmost}[2]{\left(#1 \thinspace \raisebox{-3pt}{$\xrightarrow[]{≤}$}\thinspace #2\right)}

\newcommand{\commentOC}[1]{{\selectlanguage{french}{\todo{OC : #1}}}}
%Or: \todo[color=green!40]

%this probably requires outdated float package, see doc KomaScript for an alternative.
% \newfloat{program}{t}{lop}
% \floatname{program}{PM}

%style is plain by default (italic text)
	%\newtheorem{definition}{Definition}
	%\newtheorem{theorem}{Theorem}
	%\newtheorem{lemma}{Lemma}
%\ifdefined\theorem\else
%\newtheorem{theorem}{\iflanguage{english}{Theorem}{Théorème}}
%\fi

\theoremstyle{remark}
%	\newtheorem{examplex}{Example}
	\newtheorem{remarkx}{Remark}

%trickery allowing use of \qedhere and the like.
\newenvironment{remark}{
	\pushQED{\qed}\renewcommand{\qedsymbol}{$\triangle$}\remarkx
}{
	\popQED\endremarkx
}

%which line breaks are chosen: accept worse lines, therefore reducing risk of overfull lines. Default = 200
\tolerance=2000
%accept overfull hbox up to...
\hfuzz=2cm
%reduces verbosity about the bad line breaks
\hbadness 5000
%sloppy sets tolerance to 9999
\apptocmd{\sloppy}{\hbadness 10000\relax}{}{}

% WRITING
%\newcommand{\ie}{i.e.\@\xspace}%to try
%\newcommand{\eg}{e.g.\@\xspace}
%\newcommand{\etal}{et al.\@\xspace}
\newcommand{\ie}{i.e.\ }
\newcommand{\eg}{e.g.\ }
\newcommand{\mkkOK}{{\color{green}\text{\fontspec{FreeSerif.otf}✓}}}%\checkmark%
\newcommand{\mkkNONE}{\emptyset}%ding{53}requires pifont?%\color{green}{\checkmark}
\newcommand{\mkkNO}{{\color{red}\text{\fontspec{FreeSerif.otf}✗}}}
%\newcommand{\mkkOK}{\checkmark}%\color{green}{\checkmark}
%\newcommand{\mkkREQ}{\ding{53}}%requires pifont?%\color{green}{\checkmark}
%\newcommand{\mkkNO}{}%\text{\color{red}{\textsf{X}}}

\makeatletter
\newcommand{\boldor}[2]{%
	\ifnum\strcmp{\f@series}{bx}=\z@
		#1%
	\else
		#2%
	\fi
}
\newcommand{\textstyleElProm}[1]{\boldor{\MakeUppercase{#1}}{\textsc{#1}}}
\makeatother
\newcommand{\electre}{\textstyleElProm{Électre}\xspace}
\newcommand{\electreIv}{\textstyleElProm{Électre Iv}\xspace}
\newcommand{\electreIV}{\textstyleElProm{Électre IV}\xspace}
\newcommand{\electreIII}{\textstyleElProm{Électre III}\xspace}
\newcommand{\electreTRI}{\textstyleElProm{Électre Tri}\xspace}
% \newcommand{\utadis}{\texorpdfstring{\textstyleElProm{utadis}\xspace}{UTADIS}}
% \newcommand{\utadisI}{\texorpdfstring{\textstyleElProm{utadis i}\xspace}{UTADIS I}}

%TODO
% \newcommand{\textstyleElProm}[1]{{\rmfamily\textsc{#1}}} 


\input{preamble/draw}
\tikzset{profile matrix/.style={
	matrix of math nodes, column sep=3mm, row sep=2mm, nodes={inner sep=0.5mm, anchor=base}
}}
\tikzset{rank-profile matrix/.style={
	matrix of math nodes, column sep=3mm, row sep=2mm, nodes={anchor=base}, column 1/.style={nodes={inner sep=0.5mm}}, row 1/.style={nodes={inner sep=0.5mm}}
}}
\tikzset{rank-vector/.style={
	draw, rectangle, inner sep=0, outer sep=1mm
}}
\tikzset{isolated rank-vector/.style={
	draw, matrix of math nodes, column sep=3mm, inner sep=0, matrix anchor=base, nodes={anchor=base, inner sep=.33em}, ampersand replacement=\&
}}
\newcommand{\isolatedrankvector}[2]{%
		\begin{tikzpicture}[baseline]
			\path node[isolated rank-vector] {#1\&#2\\};
		\end{tikzpicture}%
}
\newcommand{\isolatedrankvectorinarray}[2]{%
		\begin{tikzpicture}[baseline]
			\path node[isolated rank-vector] {#1\&#2\\};
			\path (current bounding box.south west) ++(0, -0.33em);
		\end{tikzpicture}%
}


\input{preamble/acronyms}

%\setbeamertemplate{headline}[singleline]
%\setbeamertemplate{footline}[authortitle]

\title[Eliciting a Voting Rule]{Eliciting a Suitable Voting Rule via Rank-Vectors}
\subject{Social choice theory}
\keywords{preference elicitation, voting rules}
\author[Olivier Cailloux]{\emph{Olivier Cailloux} \and Ulle Endriss}
\institute[LAMSADE]{LAMSADE, Université Paris-Dauphine \& ILLC, University of Amsterdam}
\date{\formatdate{21}{2}{2017}}

\begin{document}
\bibliographystyle{apalike}

\begin{frame}[plain]
	\tikz[remember picture,overlay]{
		\path (current page.south west) node[anchor=south west, inner ysep=0] (lamsade) {
			\includegraphics[height=1cm]{LAMSADE95.jpg}
		};
%		\path (lamsade.south east) ++ (0, 1mm) node[anchor=south west, inner ysep=0] (dauphine) {
%			\includegraphics[height=9mm]{Dauphine.jpg}
%		};
%		\path (dauphine.south east) node[anchor=south west, inner sep=0] (psl) {
%			\includegraphics[height=1cm]{PSL.png}
%		};
		\path (current page.south east) node[anchor=south east, inner sep=0] {
			\includegraphics[height=1cm]{illc_cropped.pdf}
		};
	}
	\titlepage
\end{frame}
\addtocounter{framenumber}{-1}

\section{Context}
\subsection{Our goal}
\begin{frame}
	\frametitle{Introduction}
	
	\begin{block}{Context}
	\begin{itemize}
		\item A \emph{committee} (a group of decision makers)
		\begin{itemize}
			\item a panel attributing a research price
			\item a management committee
		\end{itemize}
		\item Recurring decisions
		\item A decision is taken using a voting rule
		\item Voting rule: a systematic way of aggregating different opinions and decide
	\end{itemize}
	\end{block}
	\begin{block}{Our goal}
		We want to help the committee choose a suitable voting rule.
	\end{block}
\end{frame}

\begin{frame}[fragile]
	\frametitle{Voting rule}

	\begin{block}{Input}
		\begin{itemize}
			\item A set of possible alternatives (options) $\allalts$
			\item Each voter $i \in N$ has a linear order of preference over $\allalts$
			\item A profile $\prof$ associates each $i$ to such an order.
		\end{itemize}
	\end{block}
	\vfill
	\begin{minipage}[c]{(\linewidth - 3mm) * \real{0.45}}
		\vspace{-5mm}
		\begin{block}{Voting rule}
			Associates to each profile $\prof$ winning alternatives $A \subseteq \allalts$.
		\end{block}
	\end{minipage}\hspace{3mm}%
	\begin{minipage}[c]{(\linewidth - 3mm) * \real{0.55}}
		\mbox{}
		\hfill
		\begin{tikzpicture}
			\path node[profile matrix] (profile) {
				R_1&
				R_2
				\\
				| (profile11) | a&
				b
				\\
				b&
				a
				\\
				c&
				| (profile32) | c
				\\
			};
			\path ($(profile.south west)!.5!(profile.south east)$) ++ (0, -5mm) node {$\prof$};
			
			\path node[draw, rectangle, fit=(profile11) (profile32), outer xsep=2mm, outer ysep=1mm] (justprofile) {};
			\path (justprofile.east) ++ (2.5cm, 0) node[inner sep=0] (winners) {\mbox{} $\Set{a, b}$};
			\path[draw, ->] (justprofile.east) to[bend left=35] node[anchor=south] {$f$} (winners.west);

			\path[draw, decorate, decoration={brace, mirror}] (justprofile.south west) -- (justprofile.south east);
		\end{tikzpicture}
	\end{minipage}
\end{frame}

\begin{frame}
	\frametitle{Our goal}
	
	Making decisions involves two steps.
	\begin{enumerate}
		\item Establish a constitution: choose a voting rule.
		\item Solve a decision problem: apply the voting rule.
	\end{enumerate}
	
	\begin{block}{Our goal}
		We focus on step 1: help the committee choose a voting rule.
		\begin{itemize}
			\item Class of functions $\allrules$ (the set of all voting rules)
			\item Preference elicitation in order to choose a function $f \in \allrules$.
			\item We want to ask \emph{simple} questions: example-based.
		\end{itemize}
	\end{block}
\end{frame}

\subsection{General approach}
\begin{frame}[fragile]
	\frametitle{A naïve attempt}
	
	\begin{block}{A first attempt}
		Simply give a profile $\prof$ and ask for $f(\prof)$. Then iterate.
	\end{block}
\vspace{-4mm}
	\begin{center}
		\begin{tikzpicture}
			\path node[profile matrix] (profile) {
				R_1&
				R_2
				\\
				| (profile11) | a&
				c
				\\
				b&
				b
				\\
				c&
				| (profile32) | a
				\\
			};
			\path ($(profile.south west)!.5!(profile.south east)$) ++ (0, -5mm) node {$\prof$};
			
			\path node[draw, rectangle, fit=(profile11) (profile32), outer xsep=2mm, outer ysep=1mm] (justprofile) {};
			\path (justprofile.east) ++ (2.5cm, 0) node[inner sep=0] (winners) {\mbox{} $\Set{?}$};
			\path[draw, ->] (justprofile.east) to[bend left=35] node[anchor=south] {$f$} (winners.west);

			\path[draw, decorate, decoration={brace, mirror}] (justprofile.south west) -- (justprofile.south east);
		\end{tikzpicture}
	\end{center}\vspace{-2mm}
	\begin{itemize}
		\item Completely general: all functions in $\allrules$ can be reached.
	\end{itemize}
	\emph{But…} 
	\begin{itemize}
		\item One question brings very little information.
		\item Questions may be difficult to answer.
	\end{itemize}
\end{frame}

\begin{frame}
	\frametitle{General idea}
	
	\begin{itemize}
		\item Ask \emph{good} (informative, example-based) questions.
		\item Restrict the class of a priori acceptable functions to $\allrules' \subset \allrules$.
	\end{itemize}
\end{frame}

\begin{frame}
	\frametitle{Outline}
	\tableofcontents[hideallsubsections]
\end{frame}

\AtBeginSection{
	\begin{frame}
		\frametitle{Outline}
		\tableofcontents[currentsection, hideallsubsections]
	\end{frame}
}

\section{Asking good questions}
\subsection{Rank-profiles}
\begin{frame}[fragile]
	\frametitle{A different view of a profile}
	
	\begin{itemize}
		\item We want to ask more informative questions about $f$
		\item We look at profiles under a different angle
		\item A \emph{rank-vector} maps voters to ranks, $x: N \rightarrow [1, m]$
		\item All rank-vectors: $\rvspace$
	\end{itemize}
	
	\begin{center}
	\begin{tikzpicture}
			\path node[profile matrix] (profile) {
				R_1&
				R_2
				\\
				| (profile11) | a&
				a
				\\
				b&
				c
				\\
				c&
				| (profile32) | b
				\\
			};
		\path[draw, decorate, decoration={brace, mirror}] (profile.south west) -- (profile.south east);
		\path ($(profile.south west)!.5!(profile.south east)$) ++ (0, -5mm) node {$\prof$};
		
		\path (profile.north east) ++ (1.5cm, 0) node[rank-profile matrix, anchor=north west] (rank-profile) {
			&
			| (header start) | 1&
			| (header end) | 2
			\\
			| (alts start) | a&
			| (rv1 start) | 1&
			| (rv1 end) | 1
			\\
			b&
			| (rv2 start) | 2&
			| (rv2 end) | 3
			\\
			| (alts end) | c&
			| (rv3 start) | 3&
			| (rv3 end) | 2
			\\
		};
		\path node[draw, ellipse, dotted, inner sep=0, fit=(header start.north west) (header end.south east)] (N) {};
		\path (N.north) node[anchor=south, inner sep=1mm] {$N$};
		\path node[draw, ellipse, dotted, inner sep=0, fit=(alts start.north west) (alts end.south east)] (A) {};
		\path (A.west) node[anchor=east, inner sep=1mm] {$\allalts$};
		\foreach \i/\a in {1/a, 2/b, 3/c} {
			\path node[rank-vector, fit=(rv\i\space start.north west) (rv\i\space end.south east)] (rv\i) {};
			\path[<-, draw] (rv\i.east) -- ++(0.3cm, 0) node[anchor=west] {$\rprof_{\prof}(\a) \in [1, m]^N$};
		}
		\path[draw, decorate, decoration={brace, mirror, raise=2mm}] (rank-profile.south west) -- (rank-profile.south east);
		\path ($(rank-profile.south west)!.5!(rank-profile.south east)$) ++ (0, -7mm) node {$\rprof_{\prof}$};
	\end{tikzpicture}
	\end{center}
\end{frame}

\begin{frame}
	\frametitle{Rank-profile function}
	\begin{itemize}
		\item Rank-profile $\rprof \in {\left(\rvspace\right)}^\allalts$ maps alternatives to rank-vectors
		\item To each profile $\prof$ corresponds a \emph{rank-profile} $\rprof_\prof$
		\item Voting rule $f$ maps $\prof$ to $\alts \subseteq \allalts$
		\item Rank-profile voting rule $\frp$ maps $\rprof$ to $\alts \subseteq \allalts$
		\item Rank-profile voting rule $\frp$ corresponds to voting rule $f$ iff $\frp(\rprof_\prof) = f(\prof)$
	\end{itemize}
\end{frame}

\begin{frame}[fragile]
	\frametitle{Rank-profiles correspond to \emph{some} combinations of rank-vectors}
	\begin{itemize}
		\item Some sets of rank-vectors do \emph{not} form a rank-profile
		\item We assume preferences are strict
		\item Thus, for a given voter: ranks must be all different
	\end{itemize}
	Not a rank-profile:\\
	\begin{tikzpicture}
		\path node[matrix of math nodes, column sep=3mm, row sep=2mm, nodes={anchor=base}, anchor=north west] (rank-profile) {
			| (rv1 start) | 1&
			| (rv1 end) | 1
			\\
			| (rv2 start) | 2&
			| (rv2 end) | 3
			\\
			| (rv3 start) | 2&
			| (rv3 end) | 2
			\\
		};
		\foreach \i/\a in {1/a, 2/b, 3/c} {
			\path node[rank-vector, fit=(rv\i\space start.north west) (rv\i\space end.south east)] (rv\i) {};
		}
	\end{tikzpicture}
\end{frame}

\begin{frame}
	\frametitle{Symmetries of rank-profile functions}
	\begin{itemize}
		\item A rule is \emph{neutral} iff it treats the alternatives equally:
		\item after renaming alternatives, $f$ selects the renamed alternatives
		\item In that case, $\frp$ only requires a \emph{set} of rank-vectors: $\frp(x^1 = \isolatedrankvector{1}{2}, x^2 = \isolatedrankvector{2}{1}) = …$
	\end{itemize}
	\begin{itemize}
		\item A rule satisfies \emph{anonymity} iff it treats the voters equally:
		\item renaming the voters does not change the winners
		\item No similar simplification of the input of $\frp$
	\end{itemize}
\end{frame}

\begin{frame}[fragile]
	\frametitle{Condorcet property}
	\begin{definition}[Condorcet property]
	\begin{itemize}
		\item A rank-profile voting rule satisfies Condorcet iff it picks the Condorcet winner if it exists
		\item $x^1$ beats $x^2$ iff more than half of the positions satisfy $x^1_i < x^2_i$
		\item $x$ is a Condorcet winner in $\rprof$ iff it beats all other $x' \in \rprof$
	\end{itemize}
	\end{definition}
	\begin{example}[Condorcet with 3 voters, 3 alternatives]
		\centering
		\begin{tikzpicture}[baseline=(rv2 start.base)]
			\path node[rank-profile matrix] (rank-profile) {
				& 1	& 2	& 3
				\\
				a &
				| (rv1 start) | 1 & 2 &
				| (rv1 end) | 2
				\\
				b &
				| (rv2 start) | 2 & 3 &
				| (rv2 end) | 1
				\\
				c &
				| (rv3 start) | 3 & 1 &
				| (rv3 end) | 3
				\\
			};
			\foreach \i/\a in {1/a, 2/b, 3/c} {
				\path node[rank-vector, fit=(rv\i\space start.north west) (rv\i\space end.south east)] (rv\i) {};
			}
		\end{tikzpicture} $ ⇒ $ ? \pause $a$
	\end{example}
\end{frame}
%Many Condorcet rules are not WOB rules

\begin{frame}
	\frametitle{Informational view about profiles}
	\begin{itemize}
		\item \citet{sen_weights_1977}
		\item Voter $i$ has evaluation function $W_i: \allalts → \R$
		\item Social welfare functional: associates $\{W_i\}$ to ranking over $\allalts$
		\item Subject to invariance requirement
		\item Example: changing $\{W_i\}$ but respecting order does not change the output
	\end{itemize}
\end{frame}

\subsection{Using a weak-order}
\begin{frame}[fragile]
	\frametitle{Representing the preferences of the committee}
	
	\begin{itemize}
		\item We can now ask for the preference status of, e.g.,\\ \isolatedrankvector{1}{3} versus \isolatedrankvector{2}{2}
		\item Sets of such questions permit to identify a voting rule
		\item \emph{Assuming} the committee reasons in a specific way
	\end{itemize}
	\begin{itemize}
		\item We assume the committee can answer each such question
		\item With one of ${>, \sim, <}$
		\item Meaning: when $x^1 > x^2$, the voting rule must select $x^1$ rather than $x^2$ if both are present {\tiny (and similarly for $x^1 < x^2$)}
		\item The preference ${\pglobeq} = {>} ∪ {\sim}$ of the committee over rank-vectors is transitive
	\end{itemize}
\end{frame}

\begin{frame}
	\frametitle{Weak-order based rules}
	\begin{definition}[Weak-order based rules]
		\begin{itemize}
			\item $\pglobeq$ a weak-order (transitive, reflexive, connected) over $\rvspace$.
			\item The rule $f_\pglobeq$, at $\rprof$, selects those alternatives having maximal rank-vectors in $\rprof$ according to $\pglobeq$.
		\end{itemize}
		A rule $f$ is \emph{weak-order based} (WOB) if there exists $\pglobeq$ st $f = f_\pglobeq$.
	\end{definition}
\end{frame}

\begin{frame}
	\frametitle{Example of a WOB rule}
	\begin{example}[A WOB rule]
		\begin{minipage}{0.45\columnwidth}
			\centering
			${\pglobeq} = {}$\fbox{%
				\begin{tikzpicture}[baseline=(13)]
\setlength{\GraphsNodeSep}{5mm}
					\path node[isolated rank-vector, matrix anchor=east, outer sep=0, inner sep=0] (11) {1\&1\\};
					\path (11.south) ++(-2mm, -1\GraphsNodeSep) node[isolated rank-vector, matrix anchor=north east] (12) {1\&2\\};
					\path (11.south) ++(2mm, -1\GraphsNodeSep) node[isolated rank-vector, matrix anchor=north west] (21) {2\&1\\};
					\path (12.south -| 11) ++(0, -1\GraphsNodeSep) node[isolated rank-vector, matrix anchor=north, inner sep=0] (22) {2\&2\\};
					\path (22.south) ++(-2mm, -1\GraphsNodeSep) node[isolated rank-vector, matrix anchor=north east, inner sep=0] (13) {1\&3\\};
					\path (22.south) ++(2mm, -1\GraphsNodeSep) node[isolated rank-vector, matrix anchor=north west, inner sep=0] (31) {3\&1\\};
					\path (13.south) ++(0, -1\GraphsNodeSep) node[isolated rank-vector, matrix anchor=north, inner sep=0] (23) {2\&3\\};
					\path (31.south) ++(0, -1\GraphsNodeSep) node[isolated rank-vector, matrix anchor=north, inner sep=0] (32) {3\&2\\};
					\path (23.south -| 22) ++(0, -1\GraphsNodeSep) node[isolated rank-vector, matrix anchor=north, inner sep=0] (33) {3\&3\\};
					\path[->, draw] (11) -- (12);
					\path[->, draw] (11) -- (21);
					\path[->, draw] (12) -- (22);
					\path[->, draw] (21) -- (22);
					\path[->, draw] (22) -- (13);
					\path[->, draw] (22) -- (31);
					\path[->, draw] (13) -- (23);
					\path[->, draw] (13) -- (32);
					\path[->, draw] (31) -- (23);
					\path[->, draw] (31) -- (32);
					\path[->, draw] (23) -- (33);
					\path[->, draw] (32) -- (33);
				\end{tikzpicture}%
			}
		\end{minipage}%
		\begin{minipage}{0.55\columnwidth}
			\centering
			$\begin{array}{lcl}
				& \rprof_1	& f_\pglobeq(\rprof_1)\\
				\cmidrule{2-3}
				a &\isolatedrankvectorinarray{1}{2} & \mkkOK\\
				b &\isolatedrankvectorinarray{2}{1} & \mkkOK\\
				c &\isolatedrankvectorinarray{3}{3} &\\
			\end{array}$
			$\begin{array}{lcl}
				& \rprof_2	& f_\pglobeq(\rprof_2)\\
				\cmidrule{2-3}
				a &\isolatedrankvectorinarray{1}{3} & \\
				b &\isolatedrankvectorinarray{2}{2} & \mkkOK\\
				c &\isolatedrankvectorinarray{3}{1} &\\
			\end{array}$
		\end{minipage}
	\end{example}
\end{frame}

\subsection{Using preorders}
\begin{frame}
	\frametitle{Incomplete question sets}
	
	\begin{itemize}
		\item We do not want to ask every possible questions!
		\item Can we get away with only \emph{some} answers?
	\end{itemize}
	\begin{definition}[Robust rules]
		\begin{itemize}
			\item $\ppglobeq$ a \emph{preorder} (transitive, reflexive) over $\rvspace$.
			\item Look at all weak-orders $\pglobeq$ extending $\ppglobeq$.
			\item The \emph{robust} rule $F_\ppglobeq$, at $\prof$, selects those alternatives winning in \emph{some} $f_\pglobeq$ (for some $\pglobeq$ extension of $\ppglobeq$).
		\end{itemize}
		A rule $f$ is \emph{robust} if there exists $\ppglobeq$ st $f = F_\ppglobeq$.
	\end{definition}
\end{frame}

\begin{frame}
	\frametitle{Example of a robust rule}
	\begin{example}[A robust rule]
		\begin{minipage}{0.55\columnwidth}
			\centering
			${\ppglobeq} = {}$\fbox{%
				\begin{tikzpicture}[baseline=(current bounding box)]
\setlength{\GraphsNodeSep}{11mm}
					\path node[isolated rank-vector, matrix anchor=east, outer sep=0, inner sep=0] (13) {1\&3\\};
					\path (13.south) ++(0, -1\GraphsNodeSep) node[isolated rank-vector, matrix anchor=north] (41) {4\&1\\};
					\path (41.south) ++(-8mm, -1\GraphsNodeSep) node[isolated rank-vector, matrix anchor=north, inner sep=0] (14) {1\&4\\};
					\path (41.south) ++(8mm, -1\GraphsNodeSep) node[isolated rank-vector, matrix anchor=north, inner sep=0] (33) {3\&3\\};
					\path (33.east) ++(4mm, 0) node[isolated rank-vector, matrix anchor=west, inner sep=0] (34) {3\&4\\};
					\path ($(13 -| 34)!.5!(41 -| 34)$) node[isolated rank-vector, matrix anchor=center, inner sep=0] (22) {2\&2\\};
					\path[->, draw] (13) -- (41);
					\path[->, draw] (41) -- (14);
					\path[->, draw] (41) -- (33);
					\path[->, draw] (22) -- (34);
				\end{tikzpicture}%
			}
		\end{minipage}%
		\begin{minipage}{0.45\columnwidth}
			\centering
			$\begin{array}{lcl}
				& \rprof	&F_\ppglobeq(\rprof)\\
				\cmidrule{2-3}
				a &\isolatedrankvectorinarray{1}{3} & \mkkOK\\
				b &\isolatedrankvectorinarray{2}{2} & \mkkOK\\
				c &\isolatedrankvectorinarray{3}{4} &\\
				d &\isolatedrankvectorinarray{4}{1} &\\
			\end{array}$
		\end{minipage}
	\end{example}
\end{frame}

\section{Restrict the class of functions}
\subsection{WOB rules}
\begin{frame}
	\frametitle{Our restriction over possible functions}
	
	\begin{itemize}
		\item We assume the committee reasons in some specific way
		\item Restricts the class of rules
		\item Bad news: we are not fully general any more
		\item Good news: we have restricted our class of functions
	\end{itemize}
\end{frame}

\begin{frame}
	\frametitle{The WOB class}
	\begin{itemize}
		\item The committee has a weak-order $\pglobeq$ over rank-vectors “in mind”
		\item This represents a WOB rule $f_\pglobeq$
		\item $WOB = \Set{f_\pglobeq, \pglobeq \text{ a weak-order over } \rvspace}$ instead of $\allrules$
	\end{itemize}
\end{frame}

\begin{frame}
	\frametitle{WOB rules are neutral}
	\begin{itemize}
		\item If $f$ is WOB, $f$ is neutral
		\item Because $f_\pglobeq$ selects those alternatives with highest rank-vectors
		\item Thus, we care only about the set of rank-vectors as input of $f$
		\item And the rank-vector it selects
	\end{itemize}
\end{frame}

\begin{frame}
	\frametitle{How to be a WOB rule?}
	$f$ is a WOB rule iff $f$:
	\begin{itemize}
		\item Assigns a score $s(x) \in \R$ to each rank-vector $x \in \rvspace$
		\item Selects the rank-vectors having highest scores
	\end{itemize}
\end{frame}

\begin{frame}
	\frametitle{Scoring rules are WOB rules}
	\begin{itemize}
		\item Every scoring rule (e.g. Borda) is a WOB rule
		\item $s(x)$ is the sum of the partial-scores $s_r(i)$ of individual components of $x$
		\item Score of 
		\begin{tikzpicture}[baseline]
			\path node[isolated rank-vector] {1\&3\&4\\};
		\end{tikzpicture}
		$ = s_r(1) + s_r(3) + s_r(4)$
	\end{itemize}
\end{frame}

\begin{frame}
	\frametitle{Indifference to permutation is sufficient for anonymity}
	\begin{itemize}
		\item Assume $\pglobeq$ is indifferent to permutations of $x$
		\item E.g. \isolatedrankvector{1}{3} {\sim} \isolatedrankvector{3}{1}
		\item It follows that $f_\pglobeq$ satisfies anonymity:
		\item Permuting the voters permutes all rank-vectors
		\item $f$ must still select the same (reordered) rank-vectors
	\end{itemize}
\end{frame}

\begin{frame}
	\frametitle{Weak-orders and WOB rules}
	\begin{itemize}
		\item Relationship between $\pglobeq$ and $f_\pglobeq$ may be counter-intuitive!
		\item Is indifference to permutation in $\pglobeq$ required for anonymity of $f_\pglobeq$?
	\end{itemize}
	\begin{example}[A weak-order yielding a neutral WOB rule]
		\centering
		${\pglobeq} = {}$\fbox{%
			\begin{tikzpicture}[baseline=(13)]
\setlength{\GraphsNodeSep}{2mm}
				\path node[isolated rank-vector, matrix anchor=east, outer sep=0, inner sep=0] (11) {1\&1\\};
				\path (11.south) ++(-2mm, -1\GraphsNodeSep) node[isolated rank-vector, matrix anchor=north east] (12) {1\&2\\};
				\path (11.south) ++(2mm, -1\GraphsNodeSep) node[isolated rank-vector, matrix anchor=north west] (21) {2\&1\\};
				\path (12.south -| 11) ++(0, -1\GraphsNodeSep) node[isolated rank-vector, matrix anchor=north, inner sep=0] (22) {2\&2\\};
				\path (22.south) ++(-2mm, -1\GraphsNodeSep) node[isolated rank-vector, matrix anchor=north east, inner sep=0] (13) {1\&3\\};
				\path (22.south) ++(2mm, -1\GraphsNodeSep) node[isolated rank-vector, matrix anchor=north west, inner sep=0] (31) {3\&1\\};
				\path (13.south -| 22) ++(0, -1\GraphsNodeSep) node[isolated rank-vector, matrix anchor=north, inner sep=0] (23) {2\&3\\};
				\path (23.south) ++(2mm, -1\GraphsNodeSep) node[isolated rank-vector, matrix anchor=north west, inner sep=0] (32) {3\&2\\};
				\path (23.south) ++(-2mm, -1\GraphsNodeSep) node[isolated rank-vector, matrix anchor=north east, inner sep=0] (33) {3\&3\\};
				\path[->, draw] (11) -- (12);
				\path[->, draw] (11) -- (21);
				\path[->, draw] (12) -- (22);
				\path[->, draw] (21) -- (22);
				\path[->, draw] (22) -- (13);
				\path[->, draw] (22) -- (31);
				\path[->, draw] (13) -- (23);
				\path[->, draw] (31) -- (23);
				\path[->, draw] (23) -- (33);
				\path[->, draw] (23) -- (32);
			\end{tikzpicture}%
		}
	\end{example}
\end{frame}

\begin{frame}
	\frametitle{Two weak-orders, one WOB rule}
	\begin{example}[Two “equivalent” weak-orders]
		\centering
		${\pglobeq}^1 = {}$\fbox{%
			\begin{tikzpicture}[baseline=(13)]
\setlength{\GraphsNodeSep}{2mm}
				\path node[isolated rank-vector, matrix anchor=east, outer sep=0, inner sep=0] (11) {1\&1\\};
				\path (11.south) ++(-2mm, -1\GraphsNodeSep) node[isolated rank-vector, matrix anchor=north east] (12) {1\&2\\};
				\path (11.south) ++(2mm, -1\GraphsNodeSep) node[isolated rank-vector, matrix anchor=north west] (21) {2\&1\\};
				\path (12.south -| 11) ++(0, -1\GraphsNodeSep) node[isolated rank-vector, matrix anchor=north, inner sep=0] (22) {2\&2\\};
				\path (22.south) ++(-2mm, -1\GraphsNodeSep) node[isolated rank-vector, matrix anchor=north east, inner sep=0] (13) {1\&3\\};
				\path (22.south) ++(2mm, -1\GraphsNodeSep) node[isolated rank-vector, matrix anchor=north west, inner sep=0] (31) {3\&1\\};
				\path (13.south) ++(0, -1\GraphsNodeSep) node[isolated rank-vector, matrix anchor=north, inner sep=0] (23) {2\&3\\};
				\path (31.south) ++(0, -1\GraphsNodeSep) node[isolated rank-vector, matrix anchor=north, inner sep=0] (32) {3\&2\\};
				\path (23.south -| 22) ++(0, -1\GraphsNodeSep) node[isolated rank-vector, matrix anchor=north, inner sep=0] (33) {3\&3\\};
				\path[->, draw] (11) -- (12);
				\path[->, draw] (11) -- (21);
				\path[->, draw] (12) -- (22);
				\path[->, draw] (21) -- (22);
				\path[->, draw] (22) -- (13);
				\path[->, draw] (22) -- (31);
				\path[->, draw] (13) -- (23);
				\path[->, draw] (13) -- (32);
				\path[->, draw] (31) -- (23);
				\path[->, draw] (31) -- (32);
				\path[->, draw] (23) -- (33);
				\path[->, draw] (32) -- (33);
			\end{tikzpicture}%
		}
		\hspace{3em}
		${\pglobeq}^2 = {}$\fbox{%
			\begin{tikzpicture}[baseline=(13)]
\setlength{\GraphsNodeSep}{2mm}
				\path node[isolated rank-vector, matrix anchor=east, outer sep=0, inner sep=0] (11) {1\&1\\};
				\path (11.south) ++(-2mm, -1\GraphsNodeSep) node[isolated rank-vector, matrix anchor=north east] (12) {1\&2\\};
				\path (11.south) ++(2mm, -1\GraphsNodeSep) node[isolated rank-vector, matrix anchor=north west] (21) {2\&1\\};
				\path (12.south -| 11) ++(0, -1\GraphsNodeSep) node[isolated rank-vector, matrix anchor=north, inner sep=0] (22) {2\&2\\};
				\path (22.south) ++(-2mm, -1\GraphsNodeSep) node[isolated rank-vector, matrix anchor=north east, inner sep=0] (13) {1\&3\\};
				\path (22.south) ++(2mm, -1\GraphsNodeSep) node[isolated rank-vector, matrix anchor=north west, inner sep=0] (31) {3\&1\\};
				\path (13.south -| 22) ++(0, -1\GraphsNodeSep) node[isolated rank-vector, matrix anchor=north, inner sep=0] (23) {2\&3\\};
				\path (23.south) ++(2mm, -1\GraphsNodeSep) node[isolated rank-vector, matrix anchor=north west, inner sep=0] (32) {3\&2\\};
				\path (23.south) ++(-2mm, -1\GraphsNodeSep) node[isolated rank-vector, matrix anchor=north east, inner sep=0] (33) {3\&3\\};
				\path[->, draw] (11) -- (12);
				\path[->, draw] (11) -- (21);
				\path[->, draw] (12) -- (22);
				\path[->, draw] (21) -- (22);
				\path[->, draw] (22) -- (13);
				\path[->, draw] (22) -- (31);
				\path[->, draw] (13) -- (23);
				\path[->, draw] (31) -- (23);
				\path[->, draw] (23) -- (33);
				\path[->, draw] (23) -- (32);
			\end{tikzpicture}%
		}
	\end{example}
	\begin{equation}
		f_{\pglobeq^1} = f_{\pglobeq^2}
	\end{equation}
\end{frame}

\subsubsection{Bucklin}
\begin{frame}[fragile]
	\frametitle{Bucklin}
	Bucklin: a WOB rule that is not a scoring rule
	\begin{definition}[Bucklin]
		\begin{itemize}
			\item Look at rank $r$ (starting with 1)
			\item Is there a majority for ranking an alternative at $r$ or better?
			\item Iterate, stop when found a suitable rank $r$
			\item Select those alternatives that have most persons ranking them at $r$ or better
		\end{itemize}
	\end{definition}
	\begin{center}
		\begin{tikzpicture}[baseline=(middle.base)]
			\path node[profile matrix, outer sep=2em] (profile) {
				1&2&3&4&5\\
				a&a&b&b&c\\
				| (middle) | b&c&c&c&a\\
				c&b&a&a&b\\
			};
		\end{tikzpicture} $⇒$ ? \pause $c$
	\end{center}
\end{frame}

\begin{frame}[fragile]
	\frametitle{Some WOB rules are not scoring rules}
	\begin{itemize}
		\item Bucklin is a WOB rule
		\item Proof idea: let’s build a score $s(x)$ to be \emph{minimized}
		\item $s(x) = $ rank $m_x$ required plus frac.\ missing for unanimity at $m_x$
		\item Define $m_x$ as the “median” of $x$ {\tiny lowest nb $n$ st more than half the numbers are ≤ $n$}
	\end{itemize}
	\begin{equation}
		s(x) = m_x + \frac{\# x_i > m_x}{\# x_i}
	\end{equation}
	\begin{example}[Bucklin scores]
		\centering
		\begin{tikzpicture}
			\path node[rank-profile matrix, row sep=0mm] (rank-profile) {
				& 1	& 2	& 3	& 4	& 5 & m_x & s(x)
				\\
				a &
				| (rv1 start) | 1 & 1 & 3 & 3 &
				| (rv1 end) | 2 &
				2 & 2 + \frac{2}{5}
				\\
				b &
				| (rv2 start) | 2 & 3 & 1 & 1 &
				| (rv2 end) | 3 &
				2 & 2 + \frac{2}{5}
				\\
				c &
				| (rv3 start) | 3 & 2 & 2 & 2 &
				| (rv3 end) | 1 &
				2 & 2 + \frac{1}{5}
				\\
			};
			\foreach \i/\a in {1/a, 2/b, 3/c} {
				\path node[rank-vector, fit=(rv\i\space start.north west) (rv\i\space end.south east)] (rv\i) {};
			}
		\end{tikzpicture}
	\end{example}
\end{frame}

\begin{frame}
	\frametitle{WOB compared to other classes of rules}
	\begin{itemize}
		\item Every scoring rule is a WOB rule
		\item Some WOB rules are not scoring rules
		\item Many Condorcet rules are not WOB rules
	\end{itemize}
	\begin{block}{Some relationships between classes of rules}
		Scoring $\subset$ WOB; Condorcet $\cap$ WOB $= \emptyset$ (for $n = 3k, m ≥ 4$).
	\end{block}
 \end{frame}

\subsection{Robust rules}
\begin{frame}
	\frametitle{The class of robust rules}
	
	Some robust rules are not WOB rules
	\begin{example}[A robust rule]
		\begin{minipage}{0.45\columnwidth}
			\centering
			${\ppglobeq} = {}$\fbox{%
				\begin{tikzpicture}[baseline=(current bounding box)]
\setlength{\GraphsNodeSep}{11mm}
					\path node[isolated rank-vector, matrix anchor=east, outer sep=0, inner sep=0] (13) {1\&3\\};
					\path (13.south) ++(0, -1\GraphsNodeSep) node[isolated rank-vector, matrix anchor=north] (41) {4\&1\\};
					\path (41.south) ++(-8mm, -1\GraphsNodeSep) node[isolated rank-vector, matrix anchor=north, inner sep=0] (14) {1\&4\\};
					\path (41.south) ++(8mm, -1\GraphsNodeSep) node[isolated rank-vector, matrix anchor=north, inner sep=0] (33) {3\&3\\};
					\path (33.east) ++(4mm, 0) node[isolated rank-vector, matrix anchor=west, inner sep=0] (34) {3\&4\\};
					\path ($(13 -| 34)!.5!(41 -| 34)$) node[isolated rank-vector, matrix anchor=center, inner sep=0] (22) {2\&2\\};
					\path[->, draw] (13) -- (41);
					\path[->, draw] (41) -- (14);
					\path[->, draw] (41) -- (33);
					\path[->, draw] (22) -- (34);
				\end{tikzpicture}%
			}
		\end{minipage}%
		\begin{minipage}{0.55\columnwidth}
			\centering
			$\begin{array}{lcl@{\hspace*{5mm}}cl}
				& \multicolumn{2}{c}{\hspace*{3mm}\prof_1\hspace{3mm}F_\ppglobeq(\prof_1)}	& \multicolumn{2}{c}{\hspace*{2mm}\prof_2\hspace{3mm}F_\ppglobeq(\prof_2)}\\
				\cmidrule(r{2mm}){2-3}\cmidrule{4-5}
				a &\isolatedrankvectorinarray{1}{3}
 & \mkkOK & \isolatedrankvectorinarray{1}{4} &\\
				b &\isolatedrankvectorinarray{2}{2} & \mkkOK & \isolatedrankvectorinarray{2}{2} & \mkkOK\\
				c &\isolatedrankvectorinarray{3}{4} & & \isolatedrankvectorinarray{3}{3} & \\
				d &\isolatedrankvectorinarray{4}{1} & & \isolatedrankvectorinarray{4}{1} & \mkkOK\\
			\end{array}$
		\end{minipage}
	\end{example}
 \end{frame}

\begin{frame}
	\frametitle{Robust rules compared to other classes of rules}	
	\begin{block}{Some relationships between classes of rules}
		Scoring $\subset$ WOB $\subset$ Robust
	\end{block}
\end{frame}

\section{Which questions to ask?}
\begin{frame}
	\frametitle{Which questions to ask?}
	
	\begin{itemize}
		\item Assume the committee has a weak-order $\pglobeq$ in mind
		\item We want to discover much information using few questions
		\item Different questions bring different amount of information
	\end{itemize}
	\begin{definition}[Elicitation strategy]
		An elicitation strategy tells us which question should be asked considering our current knowledge
	\end{definition}
	A strategy:
	\begin{enumerate}
		\item computes the fitness of asking about a pair of rank-vectors, for each pair
		\item chooses the fittest pair
	\end{enumerate}
	We ask $q$ questions, then compare our approximation $F_\ppglobeq$ to $f_\pglobeq$
\end{frame}

\subsection{Comparing strategies}
\begin{frame}
	\frametitle{Which strategy?}
	
	We tested three strategies
	\begin{description}[pessimistic]
		\item[optimistic] fitness of ($x, y$) proportional to the number of rank-vectors dominated by $x$ or $y$, but not both
		\item[pessimistic] a variant of the previous strategy, using the min operator rather than the sum
		\item[likelihood] fitness proportional to the likelihood of a profile occurring where both rank-vectors are possible winners {\tiny (depends on the probability distribution over profiles, we assumed impartial culture)}
	\end{description}
	We assume pareto-dominance and indifference to permutations
	\begin{block}{Comparison of strategies}
		\begin{itemize}
			\item Optimistic not better than random!
			\item Likelihood much better than pessimistic
		\end{itemize}
	\end{block}
\end{frame}

\subsection{Number of questions}
\begin{frame}
	\frametitle{Number of questions}
	
	How many questions must be asked for a useful approximation?
	\begin{itemize}
		\item Our approximation has all the true winners: $f_\pglobeq(\prof) \subseteq F_\ppglobeq(\prof)$
		\item But it may have supplementary winners
		\item We are interested in the ratio of approximated VS true winners: $\frac{\card{F_\ppglobeq(\prof)}}{\card{f_\pglobeq(\prof)}}$
		\item We average it over all profiles: $\frac{1}{\card{\profiles}} \sum_{\prof \in \profiles} \frac{\card{F_\ppglobeq(\prof)}}{\card{f_\pglobeq(\prof)}}$
	\end{itemize}
	
	For $6$ voters, $6$ alternatives, using the likelihood strategy:
	\begin{center}
		\begin{tabular}{rrr}
			\toprule
			 & \multicolumn{2}{c}{Target rule}\\
			nb q & Borda & Random $\pglobeq$\\
			\midrule
			0 & 1.9 & 2.2\\
			25 & 1.3 & 1.7\\
			99 & 1.0 & 1.3\\
			\bottomrule
		\end{tabular}
	\end{center}
\end{frame}

\section{Conclusion}
\begin{frame}
	\frametitle{Conclusion}
	
	We propose to help a committee choose a voting rule.
	\begin{itemize}
		\item We introduce a different look at a profile \citep[see also][]{sen_weights_1977}
		\item We use it to ask simple questions to elicit preferences
		\item We analyse the class of rules reachable by our questioning process
		\item A robust voting rule may be defined to give all possible winners \citep[inspired by][]{dias_aggregation/disaggregation_2002}
		\item We compare and analyse several elicitation strategies
	\end{itemize}
\end{frame}

\begin{frame}
	\frametitle{Future work}
	\begin{itemize}
		\item The committee could have a preorder in mind
		\item Or the stable part of the w-o might be a preorder
		\item Behavioural interpretation of the constraints given by the committee
		\item Further analysis of the classes WOB, Robust rules
		\item Explore approximation with robust rules more generally
		\item Better elicitation strategies with active learning techniques
	\end{itemize}
\end{frame}

\begin{frame}[plain]
	\addtocounter{framenumber}{-1}
	\begin{center}
		\huge
		\textit{Thank you for your attention!}
	\end{center}
\end{frame}

\appendix
\AtBeginSection{
}

\clearpage\pdfbookmark[2]{\refname}{\refname}
\begin{frame}[allowframebreaks]
	\frametitle{\refname}
 	\bibliography{zotero}
\end{frame}

\section{Aim}
\begin{frame}
	\frametitle{More general aim}
	\begin{itemize}
		\item Choose a rule: from axioms?
		\item Difficult to consider the implications of the axioms
		\item Incompatibilities, paradoxes…
		\item We want to help a commitee choose a voting rule
		\item Do not limit to ask which axioms are suitable
		\item We should use the power of the axiomatic analysis
		\item But leave the axioms implicit
	\end{itemize}
\end{frame}

\end{document}

\clearpage\pdfbookmark{License}{License}
\begin{frame}[plain]
	\frametitle{License}
	This presentation, and the associated \LaTeX{} code, are published under the \href{http://opensource.org/licenses/MIT}{MIT license}. Feel free to reuse (parts of) the presentation, under condition that you cite the author.
	
	Credits are to be given to \href{http://www.lamsade.dauphine.fr/~ocailloux/}{Olivier Cailloux}, Université Paris-Dauphine.
\end{frame}
\addtocounter{framenumber}{-1} 
\end{document}

\begin{frame}
	\frametitle{\subsecname}
	\begin{itemize}
		\item 
	\end{itemize}
\end{frame}

